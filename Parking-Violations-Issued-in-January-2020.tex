%%
% Copyright (c) 2017 - 2020, Pascal Wagler;
% Copyright (c) 2014 - 2020, John MacFarlane
%
% All rights reserved.
%
% Redistribution and use in source and binary forms, with or without
% modification, are permitted provided that the following conditions
% are met:
%
% - Redistributions of source code must retain the above copyright
% notice, this list of conditions and the following disclaimer.
%
% - Redistributions in binary form must reproduce the above copyright
% notice, this list of conditions and the following disclaimer in the
% documentation and/or other materials provided with the distribution.
%
% - Neither the name of John MacFarlane nor the names of other
% contributors may be used to endorse or promote products derived
% from this software without specific prior written permission.
%
% THIS SOFTWARE IS PROVIDED BY THE COPYRIGHT HOLDERS AND CONTRIBUTORS
% "AS IS" AND ANY EXPRESS OR IMPLIED WARRANTIES, INCLUDING, BUT NOT
% LIMITED TO, THE IMPLIED WARRANTIES OF MERCHANTABILITY AND FITNESS
% FOR A PARTICULAR PURPOSE ARE DISCLAIMED. IN NO EVENT SHALL THE
% COPYRIGHT OWNER OR CONTRIBUTORS BE LIABLE FOR ANY DIRECT, INDIRECT,
% INCIDENTAL, SPECIAL, EXEMPLARY, OR CONSEQUENTIAL DAMAGES (INCLUDING,
% BUT NOT LIMITED TO, PROCUREMENT OF SUBSTITUTE GOODS OR SERVICES;
% LOSS OF USE, DATA, OR PROFITS; OR BUSINESS INTERRUPTION) HOWEVER
% CAUSED AND ON ANY THEORY OF LIABILITY, WHETHER IN CONTRACT, STRICT
% LIABILITY, OR TORT (INCLUDING NEGLIGENCE OR OTHERWISE) ARISING IN
% ANY WAY OUT OF THE USE OF THIS SOFTWARE, EVEN IF ADVISED OF THE
% POSSIBILITY OF SUCH DAMAGE.
%%

%%
% This is the Eisvogel pandoc LaTeX template.
%
% For usage information and examples visit the official GitHub page:
% https://github.com/Wandmalfarbe/pandoc-latex-template
%%

% Options for packages loaded elsewhere
\PassOptionsToPackage{unicode}{hyperref}
\PassOptionsToPackage{hyphens}{url}
\PassOptionsToPackage{dvipsnames,svgnames*,x11names*,table}{xcolor}
%
\documentclass[
  paper=a4,
,captions=tableheading
]{scrartcl}
\usepackage{lmodern}
\usepackage{setspace}
\setstretch{1.2}
\usepackage{amssymb,amsmath}
\usepackage{ifxetex,ifluatex}
\ifnum 0\ifxetex 1\fi\ifluatex 1\fi=0 % if pdftex
  \usepackage[T1]{fontenc}
  \usepackage[utf8]{inputenc}
  \usepackage{textcomp} % provide euro and other symbols
\else % if luatex or xetex
  \usepackage{unicode-math}
  \defaultfontfeatures{Scale=MatchLowercase}
  \defaultfontfeatures[\rmfamily]{Ligatures=TeX,Scale=1}
\fi
% Use upquote if available, for straight quotes in verbatim environments
\IfFileExists{upquote.sty}{\usepackage{upquote}}{}
\IfFileExists{microtype.sty}{% use microtype if available
  \usepackage[]{microtype}
  \UseMicrotypeSet[protrusion]{basicmath} % disable protrusion for tt fonts
}{}
\makeatletter
\@ifundefined{KOMAClassName}{% if non-KOMA class
  \IfFileExists{parskip.sty}{%
    \usepackage{parskip}
  }{% else
    \setlength{\parindent}{0pt}
    \setlength{\parskip}{6pt plus 2pt minus 1pt}}
}{% if KOMA class
  \KOMAoptions{parskip=half}}
\makeatother
\usepackage{xcolor}
\definecolor{default-linkcolor}{HTML}{A50000}
\definecolor{default-filecolor}{HTML}{A50000}
\definecolor{default-citecolor}{HTML}{4077C0}
\definecolor{default-urlcolor}{HTML}{4077C0}
\IfFileExists{xurl.sty}{\usepackage{xurl}}{} % add URL line breaks if available
\IfFileExists{bookmark.sty}{\usepackage{bookmark}}{\usepackage{hyperref}}
\hypersetup{
  pdftitle={January 2020 Parking Violations},
  pdfauthor={justjensen.co},
  hidelinks,
  breaklinks=true,
  pdfcreator={LaTeX via pandoc with the Eisvogel template}}
\urlstyle{same} % disable monospaced font for URLs
\usepackage[margin=2.5cm,includehead=true,includefoot=true,centering,]{geometry}
\usepackage[export]{adjustbox}
\usepackage{graphicx}
\usepackage{color}
\usepackage{fancyvrb}
\newcommand{\VerbBar}{|}
\newcommand{\VERB}{\Verb[commandchars=\\\{\}]}
\DefineVerbatimEnvironment{Highlighting}{Verbatim}{commandchars=\\\{\}}
% Add ',fontsize=\small' for more characters per line
\usepackage{framed}
\definecolor{shadecolor}{RGB}{248,248,248}
\newenvironment{Shaded}{\begin{snugshade}}{\end{snugshade}}
\newcommand{\AlertTok}[1]{\textcolor[rgb]{0.94,0.16,0.16}{#1}}
\newcommand{\AnnotationTok}[1]{\textcolor[rgb]{0.56,0.35,0.01}{\textbf{\textit{#1}}}}
\newcommand{\AttributeTok}[1]{\textcolor[rgb]{0.77,0.63,0.00}{#1}}
\newcommand{\BaseNTok}[1]{\textcolor[rgb]{0.00,0.00,0.81}{#1}}
\newcommand{\BuiltInTok}[1]{#1}
\newcommand{\CharTok}[1]{\textcolor[rgb]{0.31,0.60,0.02}{#1}}
\newcommand{\CommentTok}[1]{\textcolor[rgb]{0.56,0.35,0.01}{\textit{#1}}}
\newcommand{\CommentVarTok}[1]{\textcolor[rgb]{0.56,0.35,0.01}{\textbf{\textit{#1}}}}
\newcommand{\ConstantTok}[1]{\textcolor[rgb]{0.00,0.00,0.00}{#1}}
\newcommand{\ControlFlowTok}[1]{\textcolor[rgb]{0.13,0.29,0.53}{\textbf{#1}}}
\newcommand{\DataTypeTok}[1]{\textcolor[rgb]{0.13,0.29,0.53}{#1}}
\newcommand{\DecValTok}[1]{\textcolor[rgb]{0.00,0.00,0.81}{#1}}
\newcommand{\DocumentationTok}[1]{\textcolor[rgb]{0.56,0.35,0.01}{\textbf{\textit{#1}}}}
\newcommand{\ErrorTok}[1]{\textcolor[rgb]{0.64,0.00,0.00}{\textbf{#1}}}
\newcommand{\ExtensionTok}[1]{#1}
\newcommand{\FloatTok}[1]{\textcolor[rgb]{0.00,0.00,0.81}{#1}}
\newcommand{\FunctionTok}[1]{\textcolor[rgb]{0.00,0.00,0.00}{#1}}
\newcommand{\ImportTok}[1]{#1}
\newcommand{\InformationTok}[1]{\textcolor[rgb]{0.56,0.35,0.01}{\textbf{\textit{#1}}}}
\newcommand{\KeywordTok}[1]{\textcolor[rgb]{0.13,0.29,0.53}{\textbf{#1}}}
\newcommand{\NormalTok}[1]{#1}
\newcommand{\OperatorTok}[1]{\textcolor[rgb]{0.81,0.36,0.00}{\textbf{#1}}}
\newcommand{\OtherTok}[1]{\textcolor[rgb]{0.56,0.35,0.01}{#1}}
\newcommand{\PreprocessorTok}[1]{\textcolor[rgb]{0.56,0.35,0.01}{\textit{#1}}}
\newcommand{\RegionMarkerTok}[1]{#1}
\newcommand{\SpecialCharTok}[1]{\textcolor[rgb]{0.00,0.00,0.00}{#1}}
\newcommand{\SpecialStringTok}[1]{\textcolor[rgb]{0.31,0.60,0.02}{#1}}
\newcommand{\StringTok}[1]{\textcolor[rgb]{0.31,0.60,0.02}{#1}}
\newcommand{\VariableTok}[1]{\textcolor[rgb]{0.00,0.00,0.00}{#1}}
\newcommand{\VerbatimStringTok}[1]{\textcolor[rgb]{0.31,0.60,0.02}{#1}}
\newcommand{\WarningTok}[1]{\textcolor[rgb]{0.56,0.35,0.01}{\textbf{\textit{#1}}}}

% Workaround/bugfix from jannick0.
% See https://github.com/jgm/pandoc/issues/4302#issuecomment-360669013)
% or https://github.com/Wandmalfarbe/pandoc-latex-template/issues/2
%
% Redefine the verbatim environment 'Highlighting' to break long lines (with
% the help of fvextra). Redefinition is necessary because it is unlikely that
% pandoc includes fvextra in the default template.
\usepackage{fvextra}
\DefineVerbatimEnvironment{Highlighting}{Verbatim}{breaklines,fontsize=\small,commandchars=\\\{\}}

% add backlinks to footnote references, cf. https://tex.stackexchange.com/questions/302266/make-footnote-clickable-both-ways
\usepackage{footnotebackref}
\usepackage{graphicx}
\makeatletter
\def\maxwidth{\ifdim\Gin@nat@width>\linewidth\linewidth\else\Gin@nat@width\fi}
\def\maxheight{\ifdim\Gin@nat@height>\textheight\textheight\else\Gin@nat@height\fi}
\makeatother
% Scale images if necessary, so that they will not overflow the page
% margins by default, and it is still possible to overwrite the defaults
% using explicit options in \includegraphics[width, height, ...]{}
\setkeys{Gin}{width=\maxwidth,height=\maxheight,keepaspectratio}
\setlength{\emergencystretch}{3em}  % prevent overfull lines
\providecommand{\tightlist}{%
  \setlength{\itemsep}{0pt}\setlength{\parskip}{0pt}}
\setcounter{secnumdepth}{-\maxdimen} % remove section numbering

% Make use of float-package and set default placement for figures to H.
% The option H means 'PUT IT HERE' (as  opposed to the standard h option which means 'You may put it here if you like').
\usepackage{float}
\floatplacement{figure}{H}

\usepackage{booktabs}
\usepackage{longtable}
\usepackage{array}
\usepackage{multirow}
\usepackage{wrapfig}
\usepackage{float}
\usepackage{colortbl}
\usepackage{pdflscape}
\usepackage{tabu}
\usepackage{threeparttable}
\usepackage{threeparttablex}
\usepackage[normalem]{ulem}
\usepackage{makecell}

\title{January 2020 Parking Violations}
\usepackage{etoolbox}
\makeatletter
\providecommand{\subtitle}[1]{% add subtitle to \maketitle
  \apptocmd{\@title}{\par {\large #1 \par}}{}{}
}
\makeatother
\subtitle{Parking Tickets from Washington D.C.'s Open Data Site}
\author{justjensen.co}
\date{16 Oct, 2020}



%%
%% added
%%

%
% language specification
%
% If no language is specified, use English as the default main document language.
%

\ifnum 0\ifxetex 1\fi\ifluatex 1\fi=0 % if pdftex
  \usepackage[shorthands=off,main=english]{babel}
\else
    % Workaround for bug in Polyglossia that breaks `\familydefault` when `\setmainlanguage` is used.
  % See https://github.com/Wandmalfarbe/pandoc-latex-template/issues/8
  % See https://github.com/reutenauer/polyglossia/issues/186
  % See https://github.com/reutenauer/polyglossia/issues/127
  \renewcommand*\familydefault{\sfdefault}
    % load polyglossia as late as possible as it *could* call bidi if RTL lang (e.g. Hebrew or Arabic)
  \usepackage{polyglossia}
  \setmainlanguage[]{english}
\fi



%
% for the background color of the title page
%
\usepackage{pagecolor}
\usepackage{afterpage}
\usepackage[margin=2.5cm,includehead=true,includefoot=true,centering]{geometry}

%
% break urls
%
\PassOptionsToPackage{hyphens}{url}

%
% When using babel or polyglossia with biblatex, loading csquotes is recommended
% to ensure that quoted texts are typeset according to the rules of your main language.
%
\usepackage{csquotes}

%
% captions
%
\definecolor{caption-color}{HTML}{777777}
\usepackage[font={stretch=1.2}, textfont={color=caption-color}, position=top, skip=4mm, labelfont=bf, singlelinecheck=false, justification=raggedright]{caption}
\setcapindent{0em}

%
% blockquote
%
\definecolor{blockquote-border}{RGB}{221,221,221}
\definecolor{blockquote-text}{RGB}{119,119,119}
\usepackage{mdframed}
\newmdenv[rightline=false,bottomline=false,topline=false,linewidth=3pt,linecolor=blockquote-border,skipabove=\parskip]{customblockquote}
\renewenvironment{quote}{\begin{customblockquote}\list{}{\rightmargin=0em\leftmargin=0em}%
\item\relax\color{blockquote-text}\ignorespaces}{\unskip\unskip\endlist\end{customblockquote}}

%
% Source Sans Pro as the de­fault font fam­ily
% Source Code Pro for monospace text
%
% 'default' option sets the default
% font family to Source Sans Pro, not \sfdefault.
%
\ifnum 0\ifxetex 1\fi\ifluatex 1\fi=0 % if pdftex
    \usepackage[default]{sourcesanspro}
  \usepackage{sourcecodepro}
  \else % if not pdftex
    \usepackage[default]{sourcesanspro}
  \usepackage{sourcecodepro}

  % XeLaTeX specific adjustments for straight quotes: https://tex.stackexchange.com/a/354887
  % This issue is already fixed (see https://github.com/silkeh/latex-sourcecodepro/pull/5) but the
  % fix is still unreleased.
  % TODO: Remove this workaround when the new version of sourcecodepro is released on CTAN.
  \ifxetex
    \makeatletter
    \defaultfontfeatures[\ttfamily]
      { Numbers   = \sourcecodepro@figurestyle,
        Scale     = \SourceCodePro@scale,
        Extension = .otf }
    \setmonofont
      [ UprightFont    = *-\sourcecodepro@regstyle,
        ItalicFont     = *-\sourcecodepro@regstyle It,
        BoldFont       = *-\sourcecodepro@boldstyle,
        BoldItalicFont = *-\sourcecodepro@boldstyle It ]
      {SourceCodePro}
    \makeatother
  \fi
  \fi

%
% heading color
%
\definecolor{heading-color}{RGB}{40,40,40}
\addtokomafont{section}{\color{heading-color}}
% When using the classes report, scrreprt, book,
% scrbook or memoir, uncomment the following line.
%\addtokomafont{chapter}{\color{heading-color}}

%
% variables for title and author
%
\usepackage{titling}
\title{January 2020 Parking Violations}
\author{justjensen.co}

%
% tables
%

%
% remove paragraph indention
%
\setlength{\parindent}{0pt}
\setlength{\parskip}{6pt plus 2pt minus 1pt}
\setlength{\emergencystretch}{3em}  % prevent overfull lines

%
%
% Listings
%
%


%
% header and footer
%
\usepackage{fancyhdr}

\fancypagestyle{eisvogel-header-footer}{
  \fancyhead{}
  \fancyfoot{}
  \lhead[16 Oct, 2020]{January 2020 Parking Violations}
  \chead[]{}
  \rhead[January 2020 Parking Violations]{16 Oct, 2020}
  \lfoot[\thepage]{justjensen.co}
  \cfoot[]{}
  \rfoot[justjensen.co]{\thepage}
  \renewcommand{\headrulewidth}{0.4pt}
  \renewcommand{\footrulewidth}{0.4pt}
}
\pagestyle{eisvogel-header-footer}

%%
%% end added
%%

\begin{document}

%%
%% begin titlepage
%%
\begin{titlepage}
\newgeometry{left=6cm}
\newcommand{\colorRule}[3][black]{\textcolor[HTML]{#1}{\rule{#2}{#3}}}
\begin{flushleft}
\noindent
\\[-1em]
\color[HTML]{000000}
\makebox[0pt][l]{\colorRule[6F4A8E]{1.3\textwidth}{2pt}}
\par
\noindent

{
  \setstretch{1.4}
  \vfill
  \noindent {\huge \textbf{\textsf{January 2020 Parking Violations}}}
    \vskip 1em
  {\Large \textsf{Parking Tickets from Washington D.C.'s Open Data
Site}}
    \vskip 2em
  \noindent {\Large \textsf{justjensen.co}}
  \vfill
}

\noindent
\includegraphics[width=100pt, left]{justjensen-logo.pdf}

\textsf{16 Oct, 2020}
\end{flushleft}
\end{titlepage}
\restoregeometry

%%
%% end titlepage
%%



{
\setcounter{tocdepth}{2}
\tableofcontents
\newpage
}
This notebook provides information about parking tickets in Washington
D.C.

\hypertarget{aquiring-the-data}{%
\section{Aquiring the Data}\label{aquiring-the-data}}

First, we'll need to grab data directly in R using the {[}Parking Ticket
dataset{]}'s API.

\begin{Shaded}
\begin{Highlighting}[]
\CommentTok{\# url \textless{}{-} \textquotesingle{}https://opendata.arcgis.com/datasets/009dedfbaf364905a8e25181b3490cd9\_0.geojson\textquotesingle{}}
\CommentTok{\# destination\_file \textless{}{-} \textquotesingle{}january\_2020\_dc\_parking\_tickets.geojson\textquotesingle{}}
\CommentTok{\# download.file(url, destination\_file, \textquotesingle{}curl\textquotesingle{})}
\CommentTok{\# library(rgdal)}
\CommentTok{\# sp\_parking\_tickets \textless{}{-} readOGR(\textquotesingle{}january\_2020\_dc\_parking\_tickets.geojson\textquotesingle{})}

\KeywordTok{library}\NormalTok{(tidyverse)}
\NormalTok{url \textless{}{-}}\StringTok{ \textquotesingle{}https://opendata.arcgis.com/datasets/009dedfbaf364905a8e25181b3490cd9\_0.csv\textquotesingle{}}
\NormalTok{destination\_file \textless{}{-}}\StringTok{ \textquotesingle{}january\_2020\_dc\_parking\_tickets.csv\textquotesingle{}}
\KeywordTok{download.file}\NormalTok{(url, destination\_file, }\StringTok{\textquotesingle{}curl\textquotesingle{}}\NormalTok{)}
\NormalTok{df\_violations \textless{}{-}}\StringTok{ }\KeywordTok{read\_csv}\NormalTok{(destination\_file)}
\KeywordTok{colnames}\NormalTok{(df\_violations) \textless{}{-}}\StringTok{ }\KeywordTok{tolower}\NormalTok{(}\KeywordTok{colnames}\NormalTok{(df\_violations))}
\end{Highlighting}
\end{Shaded}

The first violation in our data set is a `PARK IN OFFICIAL PARKING
PERMIT ONLY SPACE' violation.

Before going further, we'll need to do a little more work to get our
dataframe set up properly!

\begin{Shaded}
\begin{Highlighting}[]
\NormalTok{df\_violations \textless{}{-}}\StringTok{ }\NormalTok{df\_violations[}\KeywordTok{c}\NormalTok{(}\StringTok{\textquotesingle{}objectid\textquotesingle{}}\NormalTok{, }\StringTok{\textquotesingle{}issue\_date\textquotesingle{}}\NormalTok{, }\StringTok{\textquotesingle{}issuing\_agency\_code\textquotesingle{}}\NormalTok{,}
                                 \StringTok{\textquotesingle{}issuing\_agency\_name\textquotesingle{}}\NormalTok{, }\StringTok{\textquotesingle{}issuing\_agency\_short\textquotesingle{}}\NormalTok{, }\StringTok{\textquotesingle{}violation\_code\textquotesingle{}}\NormalTok{,}
                                 \StringTok{\textquotesingle{}violation\_proc\_desc\textquotesingle{}}\NormalTok{, }\StringTok{\textquotesingle{}location\textquotesingle{}}\NormalTok{, }\StringTok{\textquotesingle{}disposition\_type\textquotesingle{}}\NormalTok{,}
                                 \StringTok{\textquotesingle{}disposition\_date\textquotesingle{}}\NormalTok{, }\StringTok{\textquotesingle{}fine\_amount\textquotesingle{}}\NormalTok{, }\StringTok{\textquotesingle{}total\_paid\textquotesingle{}}\NormalTok{, }\StringTok{\textquotesingle{}latitude\textquotesingle{}}\NormalTok{, }
                                 \StringTok{\textquotesingle{}longitude\textquotesingle{}}\NormalTok{, }\StringTok{\textquotesingle{}mar\_id\textquotesingle{}}\NormalTok{, }\StringTok{\textquotesingle{}gis\_last\_mod\_dttm\textquotesingle{}}\NormalTok{)]}
\KeywordTok{cat}\NormalTok{(}\KeywordTok{paste}\NormalTok{(}\StringTok{\textquotesingle{}\#\textquotesingle{}}\NormalTok{, }\StringTok{\textquotesingle{}This is a heading for Object\textquotesingle{}}\NormalTok{, df\_violations}\OperatorTok{$}\NormalTok{objectid[}\DecValTok{1}\NormalTok{], }\StringTok{\textquotesingle{}  }\CharTok{\textbackslash{}n}\StringTok{\textquotesingle{}}\NormalTok{))}
\end{Highlighting}
\end{Shaded}

\hypertarget{creating-a-chart-of-violations}{%
\section{Creating a Chart of
Violations}\label{creating-a-chart-of-violations}}

\begin{Shaded}
\begin{Highlighting}[]
\NormalTok{df\_violations}\OperatorTok{$}\NormalTok{issue\_date \textless{}{-}}\StringTok{ }\KeywordTok{as.Date}\NormalTok{(df\_violations}\OperatorTok{$}\NormalTok{issue\_date, }\DataTypeTok{format=}\StringTok{\textquotesingle{}\%Y/\%m/\%d\textquotesingle{}}\NormalTok{)}
\NormalTok{df\_violations\_per\_day \textless{}{-}}\StringTok{ }\NormalTok{df\_violations }\OperatorTok{\%\textgreater{}\%}\StringTok{ }\KeywordTok{count}\NormalTok{(issue\_date)}
\NormalTok{plt \textless{}{-}}\StringTok{ }\KeywordTok{ggplot}\NormalTok{(df\_violations\_per\_day) }\OperatorTok{+}
\StringTok{  }\KeywordTok{geom\_hline}\NormalTok{(}\DataTypeTok{yintercept=}\DecValTok{0}\NormalTok{, }\DataTypeTok{size=}\FloatTok{0.4}\NormalTok{, }\DataTypeTok{color=}\StringTok{\textquotesingle{}\#3C3C3C\textquotesingle{}}\NormalTok{)}\OperatorTok{+}
\StringTok{  }\KeywordTok{geom\_line}\NormalTok{(}\KeywordTok{aes}\NormalTok{(}\DataTypeTok{x=}\NormalTok{issue\_date, }\DataTypeTok{y=}\NormalTok{n), }\DataTypeTok{color=}\StringTok{\textquotesingle{}\#6f4a8e\textquotesingle{}}\NormalTok{, }\DataTypeTok{alpha=}\FloatTok{0.8}\NormalTok{, }\DataTypeTok{size=}\DecValTok{1}\NormalTok{) }\OperatorTok{+}
\StringTok{  }\KeywordTok{geom\_point}\NormalTok{(}\KeywordTok{aes}\NormalTok{(}\DataTypeTok{x=}\NormalTok{issue\_date, }\DataTypeTok{y=}\NormalTok{n), }\DataTypeTok{color=}\StringTok{\textquotesingle{}\#6f4a8e\textquotesingle{}}\NormalTok{, }\DataTypeTok{alpha=}\FloatTok{0.8}\NormalTok{, }\DataTypeTok{size=}\DecValTok{2}\NormalTok{) }\OperatorTok{+}
\StringTok{  }\KeywordTok{labs}\NormalTok{(}\DataTypeTok{x=}\StringTok{\textquotesingle{}\textquotesingle{}}\NormalTok{, }\DataTypeTok{y=}\StringTok{\textquotesingle{}\textquotesingle{}}\NormalTok{, }\DataTypeTok{title=}\StringTok{\textquotesingle{}Parking Violations in Washington D.C. drop off on Weekends\textquotesingle{}}\NormalTok{,}
       \DataTypeTok{subtitle=}\StringTok{"Daily Parking Violations in January 2020 from D.C.\textquotesingle{}s Open Data Site"}\NormalTok{) }\OperatorTok{+}
\StringTok{  }\KeywordTok{scale\_x\_date}\NormalTok{(}\DataTypeTok{date\_labels=}\StringTok{\textquotesingle{}\%b \%d\textquotesingle{}}\NormalTok{, }\DataTypeTok{breaks=}\KeywordTok{seq}\NormalTok{(}\KeywordTok{as.Date}\NormalTok{(}\StringTok{\textquotesingle{}2020{-}01{-}01\textquotesingle{}}\NormalTok{), }\KeywordTok{as.Date}\NormalTok{(}\StringTok{\textquotesingle{}2020{-}01{-}31\textquotesingle{}}\NormalTok{), }\DataTypeTok{by=}\StringTok{\textquotesingle{}weeks\textquotesingle{}}\NormalTok{)) }\OperatorTok{+}
\StringTok{  }\KeywordTok{theme}\NormalTok{(}\DataTypeTok{text=}\KeywordTok{element\_text}\NormalTok{(}\DataTypeTok{size=}\DecValTok{12}\NormalTok{, }\DataTypeTok{color=}\StringTok{\textquotesingle{}\#3C3C3C\textquotesingle{}}\NormalTok{),}
        \DataTypeTok{plot.title=}\KeywordTok{element\_text}\NormalTok{(}\DataTypeTok{hjust=}\DecValTok{0}\NormalTok{, }\DataTypeTok{size=}\KeywordTok{rel}\NormalTok{(}\FloatTok{1.5}\NormalTok{), }\DataTypeTok{face=}\StringTok{\textquotesingle{}bold\textquotesingle{}}\NormalTok{),}
        \DataTypeTok{plot.subtitle =} \KeywordTok{element\_text}\NormalTok{(}\DataTypeTok{hjust=}\DecValTok{0}\NormalTok{, }\DataTypeTok{size=}\KeywordTok{rel}\NormalTok{(}\FloatTok{1.1}\NormalTok{)),}
        \DataTypeTok{plot.caption=}\KeywordTok{element\_text}\NormalTok{(}\DataTypeTok{hjust=}\DecValTok{0}\NormalTok{),}
        \DataTypeTok{plot.title.position =} \StringTok{\textquotesingle{}plot\textquotesingle{}}\NormalTok{,}
        \DataTypeTok{plot.background =} \KeywordTok{element\_rect}\NormalTok{(}\DataTypeTok{fill=}\StringTok{\textquotesingle{}\#F0F0F0\textquotesingle{}}\NormalTok{), }\DataTypeTok{axis.ticks=}\KeywordTok{element\_blank}\NormalTok{(),}
        \DataTypeTok{panel.background =} \KeywordTok{element\_rect}\NormalTok{(}\DataTypeTok{fill=}\StringTok{\textquotesingle{}\#F0F0F0\textquotesingle{}}\NormalTok{), }\DataTypeTok{panel.grid=}\KeywordTok{element\_line}\NormalTok{(}\DataTypeTok{color=}\OtherTok{NULL}\NormalTok{),}
        \DataTypeTok{panel.grid.major=}\KeywordTok{element\_line}\NormalTok{(}\DataTypeTok{color=}\StringTok{\textquotesingle{}\#d2d2d2\textquotesingle{}}\NormalTok{), }\DataTypeTok{panel.grid.minor=}\KeywordTok{element\_blank}\NormalTok{(),}
        \DataTypeTok{strip.background=}\KeywordTok{element\_blank}\NormalTok{(), }\DataTypeTok{strip.text=}\KeywordTok{element\_text}\NormalTok{(}\DataTypeTok{face=}\StringTok{\textquotesingle{}bold\textquotesingle{}}\NormalTok{),}
        \DataTypeTok{plot.margin=}\KeywordTok{unit}\NormalTok{(}\KeywordTok{c}\NormalTok{(}\DecValTok{1}\NormalTok{,}\DecValTok{1}\NormalTok{,}\DecValTok{1}\NormalTok{,}\DecValTok{1}\NormalTok{), }\StringTok{\textquotesingle{}lines\textquotesingle{}}\NormalTok{))}
\KeywordTok{print}\NormalTok{(plt)}
\end{Highlighting}
\end{Shaded}

\includegraphics{Parking-Violations-Issued-in-January-2020_files/figure-latex/unnamed-chunk-3-1.pdf}

\begin{Shaded}
\begin{Highlighting}[]
\KeywordTok{ggsave}\NormalTok{(}\StringTok{\textquotesingle{}Parking Violations in Washington DC.png\textquotesingle{}}\NormalTok{, }\DataTypeTok{plot=}\NormalTok{plt, }\DataTypeTok{type=}\StringTok{\textquotesingle{}cairo\textquotesingle{}}\NormalTok{,}
       \DataTypeTok{height=}\DecValTok{5}\NormalTok{,}\DataTypeTok{width=}\DecValTok{8}\NormalTok{,}\DataTypeTok{units=}\StringTok{\textquotesingle{}in\textquotesingle{}}\NormalTok{)}
\end{Highlighting}
\end{Shaded}

\hypertarget{creating-a-table-of-violations}{%
\section{Creating a Table of
Violations}\label{creating-a-table-of-violations}}

\begin{Shaded}
\begin{Highlighting}[]
\KeywordTok{library}\NormalTok{(kableExtra)}
\NormalTok{df\_violations}\OperatorTok{$}\NormalTok{fine\_paid \textless{}{-}}\StringTok{ }\KeywordTok{ifelse}\NormalTok{(df\_violations}\OperatorTok{$}\NormalTok{fine\_amount}\OperatorTok{==}\NormalTok{df\_violations}\OperatorTok{$}\NormalTok{total\_paid,}
                                  \DecValTok{1}\NormalTok{, }\DecValTok{0}\NormalTok{)}

\NormalTok{df\_violations}\OperatorTok{$}\NormalTok{fine\_bin \textless{}{-}}\StringTok{ }\KeywordTok{case\_when}\NormalTok{(}
\NormalTok{  df\_violations}\OperatorTok{$}\NormalTok{fine\_amount }\OperatorTok{\textless{}}\StringTok{ }\DecValTok{50} \OperatorTok{\textasciitilde{}}\StringTok{ \textquotesingle{}\textless{}$50\textquotesingle{}}\NormalTok{,}
\NormalTok{  df\_violations}\OperatorTok{$}\NormalTok{fine\_amount }\OperatorTok{\textless{}}\StringTok{ }\DecValTok{100} \OperatorTok{\textasciitilde{}}\StringTok{ \textquotesingle{}$50 {-} $99\textquotesingle{}}\NormalTok{,}
\NormalTok{  df\_violations}\OperatorTok{$}\NormalTok{fine\_amount }\OperatorTok{\textless{}}\StringTok{ }\DecValTok{200} \OperatorTok{\textasciitilde{}}\StringTok{ \textquotesingle{}$100 {-} $199\textquotesingle{}}\NormalTok{,}
\NormalTok{  df\_violations}\OperatorTok{$}\NormalTok{fine\_amount }\OperatorTok{\textgreater{}=}\StringTok{ }\DecValTok{200} \OperatorTok{\textasciitilde{}}\StringTok{ \textquotesingle{}$200+\textquotesingle{}}
\NormalTok{)}

\CommentTok{\# Preparing the final dataframe for table generation}
\NormalTok{df\_violations\_fines \textless{}{-}}\StringTok{ }\NormalTok{df\_violations }\OperatorTok{\%\textgreater{}\%}
\StringTok{  }\KeywordTok{drop\_na}\NormalTok{(fine\_amount) }\OperatorTok{\%\textgreater{}\%}
\StringTok{  }\KeywordTok{group\_by}\NormalTok{(fine\_bin) }\OperatorTok{\%\textgreater{}\%}
\StringTok{  }\KeywordTok{summarise}\NormalTok{(}\StringTok{\textquotesingle{}Tickets (thousands)\textquotesingle{}}\NormalTok{=}\KeywordTok{round}\NormalTok{(}\KeywordTok{length}\NormalTok{(fine\_paid)}\OperatorTok{/}\DecValTok{1000}\NormalTok{,}\DecValTok{1}\NormalTok{),}
            \StringTok{\textquotesingle{}Percent Paid\textquotesingle{}}\NormalTok{=}\KeywordTok{paste0}\NormalTok{(}\StringTok{\textquotesingle{}\%\textquotesingle{}}\NormalTok{,}\KeywordTok{round}\NormalTok{(}\KeywordTok{sum}\NormalTok{(fine\_paid)}\OperatorTok{/}\KeywordTok{length}\NormalTok{(fine\_paid),}\DecValTok{3}\NormalTok{)}\OperatorTok{*}\DecValTok{100}\NormalTok{)) }\OperatorTok{\%\textgreater{}\%}
\StringTok{  }\KeywordTok{gather}\NormalTok{(}\StringTok{\textquotesingle{}Tickets (thousands)\textquotesingle{}}\NormalTok{, }\StringTok{\textquotesingle{}Percent Paid\textquotesingle{}}\NormalTok{, }\DataTypeTok{key=}\StringTok{\textquotesingle{}value\_type\textquotesingle{}}\NormalTok{, }\DataTypeTok{value=}\StringTok{\textquotesingle{}value\textquotesingle{}}\NormalTok{) }\OperatorTok{\%\textgreater{}\%}
\StringTok{  }\KeywordTok{spread}\NormalTok{(}\StringTok{\textquotesingle{}fine\_bin\textquotesingle{}}\NormalTok{, }\StringTok{\textquotesingle{}value\textquotesingle{}}\NormalTok{) }\OperatorTok{\%\textgreater{}\%}
\StringTok{  }\KeywordTok{select}\NormalTok{(value\_type, }\StringTok{\textquotesingle{}\textless{}$50\textquotesingle{}}\NormalTok{, }\StringTok{\textquotesingle{}$50 {-} $99\textquotesingle{}}\NormalTok{, }\StringTok{\textquotesingle{}$100 {-} $199\textquotesingle{}}\NormalTok{, }\StringTok{\textquotesingle{}$200+\textquotesingle{}}\NormalTok{)}

\KeywordTok{colnames}\NormalTok{(df\_violations\_fines)[}\DecValTok{1}\NormalTok{] \textless{}{-}}\StringTok{ \textquotesingle{}Fine Amount\textquotesingle{}}

\NormalTok{kb \textless{}{-}}\StringTok{ }\KeywordTok{kbl}\NormalTok{(df\_violations\_fines, }\DataTypeTok{format =} \StringTok{\textquotesingle{}latex\textquotesingle{}}\NormalTok{,}
          \DataTypeTok{booktabs=}\NormalTok{T, }\DataTypeTok{digits=}\DecValTok{1}\NormalTok{, }\DataTypeTok{linesep=}\StringTok{\textquotesingle{}\textquotesingle{}}\NormalTok{, }\DataTypeTok{align=}\KeywordTok{c}\NormalTok{(}\StringTok{\textquotesingle{}lrrrr\textquotesingle{}}\NormalTok{)) }\OperatorTok{\%\textgreater{}\%}
\StringTok{  }\KeywordTok{kable\_styling}\NormalTok{(}\DataTypeTok{latex\_options =} \StringTok{\textquotesingle{}striped\textquotesingle{}}\NormalTok{) }\OperatorTok{\%\textgreater{}\%}
\StringTok{  }\KeywordTok{add\_header\_above}\NormalTok{(}\KeywordTok{c}\NormalTok{(}\StringTok{\textquotesingle{} \textquotesingle{}}\NormalTok{, }\StringTok{\textquotesingle{}Cheaper\textquotesingle{}}\NormalTok{=}\DecValTok{2}\NormalTok{, }\StringTok{\textquotesingle{}More Expensive\textquotesingle{}}\NormalTok{=}\DecValTok{2}\NormalTok{))}

\CommentTok{\# kb \textless{}{-} kbl(df\_violations\_fines,}
\CommentTok{\#             \textquotesingle{}latex\textquotesingle{}, booktabs=T, digits=2, linesep=\textquotesingle{}\textquotesingle{},}
\CommentTok{\#             col.names = c(\textquotesingle{}\textquotesingle{}, \textquotesingle{}Hours\textquotesingle{}, strftime(recent\_weeks[1], \textquotesingle{}\%Y{-}\%m{-}\%d\textquotesingle{}),}
\CommentTok{\#                           strftime(recent\_weeks[2], \textquotesingle{}\%Y{-}\%m{-}\%d\textquotesingle{}),\textquotesingle{}Perc. Diff\textquotesingle{}),}
\CommentTok{\#             align=c(\textquotesingle{}lrrrr\textquotesingle{})) \%\textgreater{}\%}
\CommentTok{\#   kable\_styling(full\_width = F, latex\_options = \textquotesingle{}striped\textquotesingle{})}
\KeywordTok{print}\NormalTok{(kb)}
\end{Highlighting}
\end{Shaded}

\begin{table}[H]
\centering
\begin{tabular}[t]{lrrrr}
\toprule
\multicolumn{1}{c}{ } & \multicolumn{2}{c}{Cheaper} & \multicolumn{2}{c}{More Expensive} \\
\cmidrule(l{3pt}r{3pt}){2-3} \cmidrule(l{3pt}r{3pt}){4-5}
Fine Amount & <\$50 & \$50 - \$99 & \$100 - \$199 & \$200+\\
\midrule
\cellcolor{gray!6}{Percent Paid} & \cellcolor{gray!6}{\%30.7} & \cellcolor{gray!6}{\%22.7} & \cellcolor{gray!6}{\%19.3} & \cellcolor{gray!6}{\%10.3}\\
Tickets (thousands) & 62.3 & 40.6 & 27.2 & 3.3\\
\bottomrule
\end{tabular}
\end{table}

\end{document}
